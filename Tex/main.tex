\documentclass[preprint,showpacs,preprintnumbers,amssymb,superscriptaddress,aps,prd,nofootinbib,11pt]{revtex4-1}

\oddsidemargin=-0.2cm
\evensidemargin=-0.2cm
\topmargin=-0.5cm
\textheight=21.5cm 
\textwidth=16.7cm 

\linespread{1.2}
\selectfont 

\usepackage{graphicx}       % Include figure files
\usepackage{dcolumn}        % Align table columns on decimal point
\usepackage{bm}             % bold math
\usepackage{psfrag}
\usepackage{tensor}
\usepackage[usenames]{color}
\definecolor{navyblue}{rgb}{0.0, 0.0, 0.5}
\usepackage[linktocpage,colorlinks=true,allcolors=navyblue]{hyperref}
\usepackage{amsmath}
\usepackage{caption}
\usepackage{subcaption}
%\usepackage{xcolor}
\usepackage{physics}

%\setlength{\textwidth}{16cm} \setlength{\hoffset}{-1.3cm}
%\setlength{\textheight}{23.5cm} \setlength{\voffset}{-2cm}

\newcommand{\bmath}[1]{\mbox{\boldmath{$#1$}}}
\newcommand{\vsp}{\vspace{0.2cm}}
\newcommand{\nn}{\nonumber}
\newcommand{\cD}{\mathcal{D}}
\newcommand{\cS}{\mathcal{T}}
\newcommand{\cF}{\mathcal{F}}
\newcommand{\cL}{\mathcal{L}}
\newcommand{\cC}{\mathcal{C}}
\newcommand{\cA}{\mathcal{A}}
\newcommand{\cW}{\mathcal{W}}
\newcommand{\cU}{\mathcal{U}}
\newcommand{\cV}{\mathcal{V}}
\newcommand{\mass}{\mathfrak{m}}
\newcommand{\beq}{\begin{equation}}
\newcommand{\eeq}{\end{equation}}
\newcommand{\rhoc}{\overline{\rho}}
\newcommand{\mbar}{\overline{m}}
\newcommand{\Pin}{P^{\ell m, h}} 
\newcommand{\Pup}{P^{\ell m, \infty}}
\newcommand{\bl}[1]{\textcolor{blue}{#1}}
\newcommand{\asm}{\mathrm{a}}   % We are using 'a' for the spin parameter and the semi-major axis. I have replaced the latter with \asm.
\newcommand{\rmin}{r_{\text{min}}}
\newcommand{\rmax}{r_{\text{max}}}


\newcommand{\sam}[1]{\textcolor{blue}{[\textbf{SD}: #1]}}
\newcommand{\ethan}[1]{\textcolor{red}{[\textbf{EJG}: #1]}}

\def\p{\partial}
\def\D{\displaystyle}
\def\I{\textit}

\begin{document}

\preprint{}

 \title{Conservative Electromagnetic Self Force}


\author{Ethan J. German}
% \email{ejgerman1@sheffield.ac.uk}
\affiliation{Consortium for Fundamental Physics, School of Mathematics and Statistics,
University of Sheffield, Hicks Building, Hounsfield Road, Sheffield S3 7RH, United Kingdom}
\date{\today}

\begin{abstract}
    This is a set of notes I wrote to understand the calculation of the Conservative self force. So far it includes a quick summary of results and a review of the method of extended homogeneous solutions to the scalar case.
\end{abstract}

\maketitle

\section{The Conservative Self-Force}
The conservative self-force is the radial component of the self force, defined \cite{TorresDolan2022}:
\begin{equation}
    \mathcal{F}_r = q F_{r\mu} u^\mu
\end{equation}
Where $u^\mu$ is a tangent vector to the particle's worldline and $F_{\mu\nu}$ is the Maxwell tensor, reconstructed from the Maxwell scalars by:
\begin{equation}
    F_{\mu \nu}=2\left[\phi_2 l_{[\mu} m_{\nu]}+\phi_0 \bar{m}_{[\mu} n_{\nu]}+\phi_1\left(n_{[\mu} l_{\nu]}+m_{[\mu} \bar{m}_{\nu]}\right)\right]+\text { c.c. }
\end{equation}
where $\{l^\mu,n^\mu, m^\mu, \bar{m}^\mu\}$ form the Newmann-Pensrose (NP) null-tetrad, and c.c. refers to all complex conjugates of preceding terms. In the case of electromagnetic self force $\phi_0$ and $\phi_2$ have been determined for particles on eccentric orbits in the Kerr spacetime outside the libration region\footnote{The libration is the region $r_\text{min} < r < r_\text{max}$ where $r_\text{min}$ and $r_\text{max}$ are the periapsis and the apoapsis of orbit respectively.} \cite{german2023adiabatic}, given as:
\beq
\phi_0 = \Delta^{-1} \sum_{\ell m n} S_{+1}^{\ell m \gamma}(\theta) e^{- i \omega_{mn} t + i m \phi}  
\begin{cases}
 \alpha_{+1}^{\infty} P_{+1}^{\infty , \ell m \omega_{mn}}(r) , & r \ge \rmax , \\
 \alpha_{+1}^{h} P_{+1}^{h , \ell m \omega_{mn}}(r) , & r \le \rmin , 
\end{cases}
\eeq
\beq
2 (r-i a \cos \theta)^2 \phi_2 = \Delta^{-1} \sum_{\ell m n} S_{-1}^{\ell m \gamma}(\theta) e^{- i \omega_{mn} t + i m \phi}  
\begin{cases}
 \alpha_{-1}^{\infty} P_{-1}^{\infty , \ell m \omega_{mn}}(r) , & r \ge \rmax , \\
 \alpha_{-1}^{h} P_{-1}^{h , \ell m \omega_{mn}}(r) , & r \le \rmin . 
\end{cases}
\eeq
where $\alpha^{\infty}_{\pm1}$ are defined in \cite{german2023adiabatic}, $S^{lm\gamma}_{-1}$ are spin weighted spheroidal harmonics and $P$'s are re scaled solutions to Teukolsy's equations.  In principle $\phi_1$ can also be found outside the libration region given we know $\phi_0$ and $\phi_2$. So what remains is to determine $\phi_0$ and $\phi_2$ inside the libration region. To do this we shall apply the \textbf{method of extended homogeneous solutions}.

\section{The Method of Extended Homogeneous Solutions}
\subsection{Review of the application to the Scalar Case \cite{Warburton_2011}}
\subsubsection{A quick Review of the Scalar Case}
Consider a scalar charged particle with charge $q$ orbiting in a Kerr Space-time in an eccentric orbit. The scalar field is obtained by solving the Klein-Gordon equation:
\begin{equation}
    \nabla^\mu \nabla_\mu \Phi = - 4\pi T
\end{equation}
where $\Phi$ is the scalar-field and $T$ is the source given by:
\begin{equation}
    T = q \int \delta^4(x^\mu - x_0^\mu) \sqrt{-g} \dd \tau
\end{equation}
This a point charge on the particle's worldline $x_0^\mu$. In the Kerr geometry one can separate the scalar equation to spheroidal harmonics and frequency modes:
\begin{equation}
    \Phi = \int \sum\limits_{l=0}^\infty \sum \limits_{-l}^{l} R^{lm\omega} S^{lm\gamma} e^{im\phi} e^{-imt} \dd \omega
\end{equation}
with $\gamma = a \omega$. This anstaz separates the equation of motion of the field into two equations, one defining the spin-weighted spheroidal harmonics $S^{lm\gamma}$ and one defining the Radial function:
\begin{equation}
    \mathcal{T} R_{lmn} = -4\pi \Delta T_{lmn}(r)
\end{equation}
where $\mathcal{T}$ is the Teukolsky radial equation operator for $s=0$ and $T_{lmn}$ is the inverse Fourier transform of the source $T$, and $R_{lmn} = R_{lm \omega_{mn}}$ with $\omega_{mn} = m \Omega_\phi + n \Omega_r$. Note that $T_{lmn}$ is zero outside of the libration region and only has contributions inside the libration region. If we re-scale the radial function to be $\psi_{lmn} (r) = r R_{lmn}(r)$ and introduce the tortise coordinate $r_*$ by defining $\dv{r_*}{r} = \frac{r^2}{\Delta}$, then the differential equation takes the nice form:
 \begin{equation}
     \dv[2]{\psi_{lmn}}{r_*}  + V_{lmn} (r)\psi_{lmn} = -\frac{4\pi \Delta}{r^3} T_{lmn}(r) =: Z_{lmn}(r) \label{eqn:radialEquation}
 \end{equation}
 where $V_{lmn}(r)$ is a vector potential given by equation (25) in \cite{Warburton_2011}. By applying the method of variation of parameters one can define a solution outside the libration region:
 \begin{equation}
     \psi_{lmn} (r) =  \left\{ \begin{aligned}
         C^-_{lmn} \psi^-_{lmn} =: \tilde{\psi}^-_{lmn} && r\leq r_\text{min}\\
         C^+_{lmn} \psi^+_{lmn} =: \tilde{\psi}^+_{lmn} && r\geq r_\text{max}\\
     \end{aligned}\right.
 \end{equation}
 where $\psi^\pm_{lmn}$ are the two homogeneous solutions to Eq.~\eqref{eqn:radialEquation}, and $C^\pm$ are the solution coefficients\footnote{These are analogous to $\alpha^\infty_{\pm1}$ and $\alpha^h_{\pm1}$ in the electric case.} defined in Eq.~(44) in \cite{Warburton_2011}.
\subsubsection{Spheroidal Projections}
In order to regularise using the standard mode-sum approach, we require that the modes are \textbf{spherical harmonic} modes as opposed to \textbf{spin-weighted spheroidal harmonic} modes. To deal with this we project the spin-weighted spheroidal harmonics $S_{\hat{l}m\gamma}(\theta)e^{im\varphi}$ onto spherical harmonics $Y_{lm}(\theta, \varphi)$ in the standard change-of-basis procedure:
\begin{equation}
    S_{\hat{l}m\gamma}(\theta) e^{im\varphi} = \sum\limits_{l=0}^\infty b^{\hat{l}}_{lm}(\gamma) Y_{lm}(\theta,\varphi)
\end{equation}
where $b^{\hat{l}}_{lm}$ are the $\gamma$-dependant coefficients of projection, which can be found in literature. Once projected onto these spherical harmonics we write $\Phi$ as a mode-sum of spherical harmonics:
\begin{equation}
    \Phi = \sum\limits_{l=0}^\infty \sum\limits_{m=-l}^l \phi_{lm}(t,r) Y_{lm}(\theta,\phi)/r
\end{equation}
where:
\begin{equation}
    \phi_{lm} = \sum\limits_{n=-\infty}^\infty \phi_{lmn}(t,r) 
\end{equation}
and:
\begin{equation}
    \phi_{lmn}(t,r) = \sum_{\hat{l} =0}^\infty b^{\hat{l}}_{lm} \psi^\text{inh}_{\hat{l}mn} e^{-i\omega t}
\end{equation}
where $\psi^\text{inh}$ is the full inhomogeneous solution to Eq.~\eqref{eqn:radialEquation}. So in theory at this stage, since we have not yet applied the method of extended homogeneous solutions we have the full solution for $\Phi$ outside the libration region. 
 \subsubsection{Applying the method of Extended Homogeneous Solutions}

The method of extended homogeneous solutions starts by extending the domain of the homogeneous solutions to the entire domain $r>2M$ by defining:
\begin{equation}
    \begin{aligned}
    \tilde{\psi}^\pm_{lmn} (r)  := C^\pm_{lmn} \psi(r)^\pm_{lmn},  && r>2M.
    \end{aligned}
\end{equation}
This is known as the \textit{Domain-extended homogeneous solution} and is analytic over its entire domain. The inhomogeneous solution is assumed to be an analytic function everywhere in the region $r>2M$ except at the position of the particle $r=r_0(\tau)$. In the region $r>r_\text{max}$ the full inhomogeneous solution is known and is identical to the domain-extended homogeneous solution $\tilde{\psi}^+$ by its definition. Thus since we have two analytical functions that are identical on $r>r_\text{max}$ then they are identical for $r>r_0(\tau)$. Note that its not on the full domain $r>2M$ since the inhomogeneous solution is not analytic along the particle's worldline creating two disconnected regions where the inhomogeneous solution is separately analytic. Similarly one can argue that the inhomogeneous solution is $\tilde{\psi}^-$ in the region $2M<r<r_0(\tau)$. So the full inhomogeneous solution is:

\begin{equation}
    \psi^\text{inh}_{lmn} = \left\{ 
    \begin{aligned}
         C^-_{lmn} \psi^-_{lmn} =: \tilde{\psi}^-_{lmn} && r\leq r_0(\tau)\\
         C^+_{lmn} \psi^+_{lmn} =: \tilde{\psi}^+_{lmn} && r\geq r_0(\tau)
    \end{aligned}
    \right.
\end{equation}

\subsection{Applying the method to the electric case}
The argument carries forward identically. We extend the domains of the homogeneous giving the maxwell scalars:
\beq
\phi_0 = \Delta^{-1} \sum_{\ell m n} S_{+1}^{\ell m \gamma}(\theta) e^{- i \omega_{mn} t + i m \phi}  
\begin{cases}
 \alpha_{+1}^{\infty} P_{+1}^{\infty , \ell m \omega_{mn}}(r) , & r \ge r_0(\tau) , \\
 \alpha_{+1}^{h} P_{+1}^{h , \ell m \omega_{mn}}(r) , & r \le r_0(\tau) , 
\end{cases}
\eeq
\beq
2 (r-i a \cos \theta)^2 \phi_2 = \Delta^{-1} \sum_{\ell m n} S_{-1}^{\ell m \gamma}(\theta) e^{- i \omega_{mn} t + i m \phi}  
\begin{cases}
 \alpha_{-1}^{\infty} P_{-1}^{\infty , \ell m \omega_{mn}}(r) , & r \ge r_0(\tau) , \\
 \alpha_{-1}^{h} P_{-1}^{h , \ell m \omega_{mn}}(r) , & r \le r_0(\tau). 
\end{cases}
\eeq

\bibliographystyle{apsrev4-1}
\bibliography{conservativeSelfForce}
\end{document}
