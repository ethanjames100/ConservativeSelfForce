\documentclass[preprint,showpacs,preprintnumbers,amssymb,superscriptaddress,aps,prd,nofootinbib,11pt]{revtex4-1}

\oddsidemargin=-0.2cm
\evensidemargin=-0.2cm
\topmargin=-0.5cm
\textheight=21.5cm 
\textwidth=16.7cm 

\linespread{1.2}
\selectfont 

\usepackage{graphicx}       % Include figure files
\usepackage{dcolumn}        % Align table columns on decimal point
\usepackage{bm}             % bold math
\usepackage{psfrag}
\usepackage{tensor}
\usepackage[usenames]{color}
\definecolor{navyblue}{rgb}{0.0, 0.0, 0.5}
\usepackage[linktocpage,colorlinks=true,allcolors=navyblue]{hyperref}
\usepackage{amsmath}
\usepackage{caption}
\usepackage{subcaption}
%\usepackage{xcolor}
\usepackage{physics}

%\setlength{\textwidth}{16cm} \setlength{\hoffset}{-1.3cm}
%\setlength{\textheight}{23.5cm} \setlength{\voffset}{-2cm}

\newcommand{\bmath}[1]{\mbox{\boldmath{$#1$}}}
\newcommand{\vsp}{\vspace{0.2cm}}
\newcommand{\nn}{\nonumber}
\newcommand{\cD}{\mathcal{D}}
\newcommand{\cS}{\mathcal{T}}
\newcommand{\cF}{\mathcal{F}}
\newcommand{\cL}{\mathcal{L}}
\newcommand{\cC}{\mathcal{C}}
\newcommand{\cA}{\mathcal{A}}
\newcommand{\cW}{\mathcal{W}}
\newcommand{\cU}{\mathcal{U}}
\newcommand{\cV}{\mathcal{V}}
\newcommand{\mass}{\mathfrak{m}}
\newcommand{\beq}{\begin{equation}}
\newcommand{\eeq}{\end{equation}}
\newcommand{\rhoc}{\overline{\rho}}
\newcommand{\mbar}{\overline{m}}
\newcommand{\Pin}{P^{\ell m, h}} 
\newcommand{\Pup}{P^{\ell m, \infty}}
\newcommand{\bl}[1]{\textcolor{blue}{#1}}
\newcommand{\asm}{\mathrm{a}}   % We are using 'a' for the spin parameter and the semi-major axis. I have replaced the latter with \asm.
\newcommand{\rmin}{r_{\text{min}}}
\newcommand{\rmax}{r_{\text{max}}}


\newcommand{\sam}[1]{\textcolor{blue}{[\textbf{SD}: #1]}}
\newcommand{\ethan}[1]{\textcolor{red}{[\textbf{EJG}: #1]}}

\def\p{\partial}
\def\D{\displaystyle}
\def\I{\textit}

\begin{document}

\preprint{}

 \title{Conservative Electromagnetic Self Force}


\author{Ethan J. German}
% \email{ejgerman1@sheffield.ac.uk}
\affiliation{Consortium for Fundamental Physics, School of Mathematics and Statistics,
University of Sheffield, Hicks Building, Hounsfield Road, Sheffield S3 7RH, United Kingdom}
\date{\today}
\maketitle

\section{The Conservative Self-Force}
The conservative self-force is the radial component of the self force, defined \cite{TorresDolan2022}:
\begin{equation}
    \mathcal{F}_r = q F_{r\mu} u^\mu
\end{equation}
Where $u^\mu$ is a tangent vector to the particle's worldline and $F_{\mu\nu}$ is the Maxwell tensor, reconstructed from the Maxwell scalars by:
\begin{equation}
    F_{\mu \nu}=2\left[\phi_2 l_{[\mu} m_{\nu]}+\phi_0 \bar{m}_{[\mu} n_{\nu]}+\phi_1\left(n_{[\mu} l_{\nu]}+m_{[\mu} \bar{m}_{\nu]}\right)\right]+\text { c.c. }
\end{equation}

where $\{l^\mu,n^\mu, m^\mu, \bar{m}^\mu\}$ form the Newmann-Pensrose (NP) null-tetrad, and c.c. refers to all complex conjugates of preceding terms. In the case of electromagnetic self force $\phi_0$ and $\phi_2$ have been determined for particles on eccentric orbits in the Kerr spacetime \cite{german2023adiabatic}. It remains to determine $\phi_1$.

To determine $\phi_1$ we can use one of the four Maxwell equations \cite{Teukolsky:1973ha}:
\begin{equation}
\begin{aligned}
(D-2 \rho) \phi_1-\left(\delta^*+\pi-2 \alpha\right) \phi_0 & =2 \pi J_l, \\
(\delta-2 \tau) \phi_1-(\Delta+\mu-2 \gamma) \phi_0 & =2 \pi J_m, 
\end{aligned}
\end{equation}
We need only solve one of these to determine $\phi_1$, the other can be used as a consistency check. 
We take the same source term as \cite{german2023adiabatic}: 
\begin{equation}
J^\mu=q \int u^\mu \delta^4\left(x-x_0(\tau)\right) d \tau
\end{equation} 
\bibliographystyle{apsrev4-1}
\bibliography{conservativeSelfForce}
\end{document}
